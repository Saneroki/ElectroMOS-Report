\section{Ricisi ved Projektarbejdet}
Gruppen har identificeret et par forskellige risici, når det kommer til udviklingen af Electroshoppens CMS system. Den største risici når det kommer til ethvert projekt, er at man sætter sine mål forhøje til de ressourcer man har til rådighed. Derfor kan man ende med et ufærdigt produkt som kunden havde en højere forventning til. Derfor er det vigtigt at lave et realistisk mål med de ressourcer gruppen, har lige i præcis i dette projekt.  Dette betyder at vi skal tage højde for manglende erfaring, begrænset tid, og meget lavt budget. Hvilket gør at det produkt vi levere ikke kan levere på samme måde som andre løsninger på markedet. 

Når man arbejder med programmering og Kodning generelt, er der altid en mulighed for at koden kan slå fejl, eller opfører sig uhensigtsmæssigt. Disse fejl håbes at kunne opdages inden for testing fasen, men da igen vi har begrænset ressourcer til rådighed, betyder det mængden af test ikke kan komme op på det ønskede niveau.

Hvis der skal ses på et forretningsniveau, kan dårlig kommunikation mellem kunden og producent, lede til et produkt kunden ikke havde forventet. Da det optimale mål ville at have kunder møder oftere end dette projektforløb betyder det at meget af produktet bliver lavet på producentens forståelse af projekt casen. 

