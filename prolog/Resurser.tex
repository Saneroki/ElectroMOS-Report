\section{Resurser}

\subsection{Forvented Indsats}
Gruppen består af 6 medlemmer, der hver, som udgangspunkt, forventes at bruge 10-12 timer om ugen, per medlem. denne mængde timer justeres alt efter status på projektet. igennem projektforløbet er der meget fastsatte deadlines, hvilket giver god mulighed for at justere arbejdsbyrden ud fra dem. der bliver primært arbejdet på projektet på dag hvor der er undervisningfri.

Mængden af timer brugt om ugen stiger også parallelt med mængden af afsluttede fag, så der trappes naturligt op på mængden af tid.


\subsection{Værktøjer}
\paragraph{Java}
Det primærer værktøj til programmering af systemet er Java, da java er det fortrukket programmerings sprog for både fra underviser og kundens side. Af samme årsag er webshoppen og CMS systemets frontend udviklet i JavaFX.

\paragraph{JavaFX}
JavaFX bruger værktøjet Scenebuilder til at preview design af GUI. dette værktøøj kan også bruges til opsætningen af designet, så der ikke er behov for at skulle kode layout attributer. 

\paragraph{Postgresql}
PostgreSQL bliver brugt til databaseprogrammering, dette er igen primært grundet kundens ønsker, men da postgres opfylder de krav, som produktet kræver, ved at være pålidlig, og relativt nemt at håndtere til fremtidig udvikling plus tilføjelser af nye databaser for de andre systemer.

\paragraph{Draw.io}
Draw.io til diagrammering. Draw.io er det værktøj der var mest erfaring med mellem gruppens medlemmer. den leverer de skabaloner der skal bruges for at opfylde UML standard, inden for diagrammer.

\paragraph{Latex}
Latex og Overleaf er blive brugt til struktur og opbygning af rapport. dette værktøjer giver gode muligheder for at få skabt en rapport struktur, der er overskuelig og nem at læse. 

\paragraph{Google Docs}
Google Docs er Googles svar på microsoft word. Google Docs har været brugt til at skrive rapport i. Inceptionsrapporten(og alt andet i projektet før det) er skrevet i Google Docs.
