%\newpage
\thispagestyle{fancy}
% Gruppearbejdet har været meget svingene i løbet af dette projekt. der startede ud med fuldtalligt gruppe på 6 medlemmer, og kørte en fornuftig arbejds fordeling igennem hele projekt etablering og inceptionsfasen.
% \\ \vspace{5mm}
% inceptionfase var det primært arbejde med struktur af CMS systemet, og næsten alle havde et input i hvordan det skulle bygges. vi afsluttede inceptions fasen med et fornuftigt resultat som alle havde arbejdet på.
% \\ \vspace{5mm}
% Sammenarbejdet mellem de resterende 3 medlemmer har været fornuftigt, og de har vist en højere pålidelighed. Men selv på dette tidspunkt var der ubalanceret arbejdsfordeling. To gruppemedlemmer har ført sig bedre fremad i det meste resterende arbejde, hvor den tredje har prøvet at følge med, men det er gået langsommere for denne trejde person.
% \\ \vspace{5mm}
% Der har været, for nogle gruppemedlemmer, personlige begivenheder og hændelser som har lammet deres bidragen til arbejdet i nogen eller længere tid. 
% \\ \vspace{5mm}
% Arbejdsmoralen var fin i starten af projektet, men for nogle er Arbejdsmoral dykket meget. Det er gået ud over de andre gruppemedlemmer og har sat spørgsmålstegn på hinandens tillid.

Formålet med inceptionsfasen og de 2 iterationer i elaborationsfasen var at få etableret en gode grundstruktur for Content Management systemet. Det lykkedes for projektet vha. Widget systemet der giver mulighed at tilføje ekstra funktionalitet til hjemmesider på modulbasis. Det vil sige, at man ved hjælp af et afkoblet system kan tilføje widgets uden at tilføje kode til de andre subsystemer. Dette gøre det meget fleksibel at videreudvikle systemet i fremtiden. Desuden er der blevet lavet et struktur, der muliggøre skift af helt uafhængige scener ved hjælp af en enumerationsklasse, der indeholder de basale informationer om alle scener. Dette gør det nemt at tilføje flere scener, der tager sig om andre specifikke opgaver.  Projektet bruger individuelle FXML filer til hver scene og til hver widget. Dette sørger for at opsætningen af brugergrænsefladen er separeret i ekstra filer for hver scene og widget der enkeltvis bliver brugt når dette er nødvendig. 

Indelingen af programmet i scener og widgets med hver deres FXML fil gør koden mere overskueligt, og giver mulighed for at tilføje nye widgets og scener med hver deres egen klasse, uden at redigere andre klasser ud over enumerationsklassen for scener. 
Inden projektstart har der været begrænset kendskab mht. JavaFX-strukturen med flere forskellige scener eller fxml filer. Dette sørgede for at at ekstra indsats af research for mulighederne og få dem til at virke i en prototype. 

Alt i alt er programmet ikke særlig brugbart i nuværende stadie, men ligger grundsten for kommende iterationer, hvor der kan lægges fokus på at gøre Pageplanner mere brugervenlig, designe widgets og tiføje flere funktioner til programmet, CMS.







