Målet var at opsætte et CMS system der gave nem mulighed for elektroshoppens ansatte, at kunne holde styr på deres hjemmeside, både ved at lave nye sider og opdatere og slette gamle sider.
CMS systemet har fået sat en standard struktur op, der giver rigtig god mulighed for at kunne arbejde videre på. Denne struktur er baseret på at vi bygger hver side op igennem widgets, som er små funktionalitet man typisk ser på hjemmesider. Disse widget består lige nu af searchbar, browser vindue, produktcontainer og campaign. På dette stadie er disse widget meget "dumme", da de ikke har alt information fra relateret systemer som PIM, DAM og ERP. men som der bliver udviklet mere i løbet af de næste faser og iterationer. Det er relativt nemt at holde både webshoppen og CMS systemet overskueligt, selvom flere widgets bliver tilføjet. Der er stadig mere komplekse widgets der skal arbejdes på, plus et par burgsmønster skal stadig realiseres.  Da dette ville være et mere omfattende projekt. Men som udgangspunkt er der en FXML og controller fil, der hver kan tilføjes funktionalitet til.

Unified Process har givet værktøjerne til nemt at arbejde videre på projektet, også selvom det nødvendigvis ikke er samme udvikler. Dog blev UP ikke 100\% fulgt da mange modeller og diagrammer ikke virkede nødvendig for et program på dette stadie. Hvis muligheden var der for at lave programmet færdig, ville der helt sikkert blive gjort flere overvejelser over hvilket værktøjer der bliver brugt. Der er enighed om at UP måske ikke er den helt rigtige metode at bruge til ligge præcis denne udvikling. Specielt når der ikke ligger mere erfaring bag ved udviklerne. Da UP kræver man har en forståelse for det system man bygger forud er det meget svært at kunne identificerer metode, kald, attributter, klasser, osv. uden at skulle iterere processen igennem, mere end tidsplanen tillader det. Alternativt kunne man bruge scrum der tillader en mere iterativ proces, der giver mulighed for at kunne omstrukturer sin implementeringer.

Et problem der er opstået undervejs var at disse widget filer både skulle kendes af CMS systemet og webshoppen. Dette betyder at widget pakken altid skal matche hinanden for at systemet kan fungerer. Den midlertidige løsning er at lave en perfekt kopi af CMS mappens widgets, og altid holde den opdateret. Dog ville det i fremtidige udvikling give mening at disse to mapper automatisk opdateret sig selv. Dette kunne gøres igennem database, eller eventuelt to programmer arbejder ud fra samme mappe. Men for at kunne udvikle noget der er optimalt, skal der bruges mere tid.
 
 
I næste fase ville det være optimalt at begynde arbejde på widget funktionalitet. Features som at kunne resize widgets ville også være brugbart og eventuelt skabelsen af nye widget.
I elaborations fasen var fokus primært at bygge et godt skelet af programmet, og via diagrammerne i analyse og design, er der skabt et fornuftigt struktur at kunne bygge ud fra.

