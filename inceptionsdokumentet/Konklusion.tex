Problemstillingen omhandler eksistensgrundlaget for systemet, der udvikles for at opfylde mange af Electroshoppens mål. Dette gøres ved at udvikle et CMS i systemlandskabet med projektnavnet ElectroMOS. Dette har til opgave at give Electroshoppen en nem måde at redigere deres hjemmeside, uden at have kendskab til programmering, ved at oprette specifikke sider som landingssider, kampagnesider, forside og redigering i brugerrettigheder af hjemmesiden. 
 
I forretningsmodellen blev der analyseret at ElectroMOS kan opfylde rigtig mange af Electroshoppens mål. Ved hjælp af ElectroMOS’s funktioner, vil det give Electroshoppen mulighed for en mere personlig kundeoplevelse, besparelser i form af fragt og lageromkostninger, og nemmere bearbejdelse af reklamer til kunder. Et forventet resultat vil være en mere ens kundeoplevelse blandt de fysiske butikker og Electroshoppens hjemmeside. Mange opgaver kunne blive automatiseret som yderligere sparer lønomkostningerne.
 
I afsnittet “Krav” specificeres der grundigt, hvad der skal til for at ElectroMOS kan opfylde de opgaver, der er beskrevet i forretningsmodellen. Ved hjælp af et detaljeret brugsmønsterdiagram og et eksempel på en brugsmønsterbeskrivelse gives et godt overblik over, hvordan sådan et system vil bruges i praksis. En pakke- og klassediagram viser opbygningen af systemet, som bliver suppleret af kravspecifikationer, der skal være opfyldt for at systemet opfylder opgaverne. 
