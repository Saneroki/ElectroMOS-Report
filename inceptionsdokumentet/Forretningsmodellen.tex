
I forretningsmodellen vil der blive beskrevet hvordan ElectroMOS vil opfylde Electroshoppens forretningsmål og dens strategi. Denne model kan ses som den strategi, man vil bruge til at implementere virksomhedens organisationsstruktur, processer og vil dermed bygge eksistensgrundlaget for produktet. Dette vil beskrive grunde til Electroshoppen at købe ElectroMOS.

Der vil følgende blive givet en beskrivelse af virksomheden, og hvordan dets struktur lægger til rette, vha. Osterwalders model. Osterwalders model har 9 byggeklodser, hvori de 4 hovedområder der stort set grundlægger en forretning på bliver beskrevet. Herved menes der altså kundekontakt. Den indeholder hvad der vil blive tilbudt kunderne mht. produkt, infrastrukturen og sidst men ikke mindst de finansielle sammenhænge, der viser hvordan virksomheden tjener penge.

\section{Værdifaktorer}
ElectroMOS kunne øge Electroshoppens værdifaktorer på mange forskellige måder. ElectroMOS kunne hjælpe med at give et mere dybt varesortiment ved at oprette specialiserede undersider for produkter og produktgrupper, der giver kunder et bedre overblik over disse. Dermed kan Electroshoppen bedre dække kundens behov for mere specialiserede varer og et dybt sortiment. Der kunne for eksempel oprettes sider, der repræsentere kendte varemærker eller repræsenterer indretning af de fysiske butikker, som dermed skaber et genkendelighed og brugervenlighed blandt kunderne.

\section{Kunder}
Electroshoppen har en meget bred målgruppe. Mulige kunder kan variere fra kunder der ikke er vant til at bruge internettet, og ikke ved hvilket produkt de vil købe til kunder der er tekniknørder og ved helt præcis, hvordan de navigerer de mest komplekse hjemmesider, og hvilke produkter de vil have til hvilken pris. En anden brugerkategori til ElectroMOS vil være Electroshoppens medarbejdere, der bruger ElectroMOS’ backend til at oprette sider, som kunden bliver præsenteret for, og sender de nye links rundt via. maillisten. 


\paragraph{Personae 1: Hygge Shopper Gudrun.}
Gudrun er 56 og elsker at gå igennem byen og shoppe i de fysiske butikker. Hun køber gaver, og kommer forbi Electroshoppens fysiske butik. Hun møder ved indgangen en iPad på en piedestal 

med et skilt: “Find din vare her”. Hun går hen til iPad’en og ser en hjemmeside hvori hun kan søge produkter. Hun navigerer nemt igennem hjemmesiden, og vælger et produkt ud. På hjemmesiden står at produktet findes i reol 4 på hylde 11 med tilvalget at betale direkte på hjemmesiden og oprettelse af kundeinformationer. Hun vælger at forlade iPad’en og samler sit produkt op ved den angivne lokation, går op til kassen for at betale der.


\paragraph{Personae 2: Preisjaeger Peter.}
Peter er 21 og studerer. Derfor har han ikke ret mange penge og vælger at købe sine produkter når de er særligt billige. Tidligere har Peter handlet hos Electroshoppen og tilvalgt at modtage nyhedsbreve. Electroshoppen har en kampagne kørende for ekstra billige batterier for mobilerne: Samsung Galaxy S3 og Samsung Gear. Peter har brug for batterier jævnligt og vælger at åbne mailen og følge linket. Dette link navigere ham direkte til en landingsside hvor han kan vælge at købe produkterne. Han har blandt andet mulighed for at finde en butik i nærheden, der sælger det, eller få det sendt direkte hjem til ham. Han vælger at få dem sendt hjem. Han opdager, at fragten koster lige så meget som batterierne. Derfor vender han tilbage til hjemmesiden og tilføjer en Samsung Galaxy i sin indkøbskurve, og vælger denne gang at hente produkterne fra en af butikkerne, der ligger på hans vej hjem fra Universitetet.


\paragraph{Personae 3: Electroshoppens ansatte Tiffany.}
Tiffany er 32 og ansat hos Electroshoppen. Hun arbejder i en af de fysiske butikker, men opretter også kampagner og landingssider til hjemmesiden. Hun er den der kommer til at arbejde primært med ElectroMOS. Når hun opdager, at der er et bestemt vare som der er blevet købt for meget af og ikke bliver solgt i de fysiske butikker, opretter hun nogle landingssider og sender et link ud til disse via mailinglisten. Når der er produkter, som Electroshoppens kæde kan sælge for en billigere pris end konkurrencen, sætter hun kampagnesider op til disse produkter og sender dem også ud via mailinglisten.


\section{Kunderelationer}
Electroshoppen har et mål om altid at have kundens shoppingoplevelse i højeste kvalitet. Dette giver tilbagevendende kunder, der er villige til at handle i Electroshoppen jævnligt. Denne service skal komme i form af positive shoppingoplevelser. Der skal ikke være tvivl omkring hvordan man køber og får fuldt udbytte af sit produkt. Når man definerer positive shoppingoplevelser, er der tale om hjælp af medarbejdere til at udvælge lige præcis det, de har brug for, for at have den tillid til at man får det rigtige produkt, når man handler i Electroshoppen. Derudover er der god support efter købet af produkt, hvor hjælp og support med produktet fortsætter. Der er reklamationsret, hvis produktet skulle gå i stykker, og muligvis fordele for fremtidige køb.

\section{Vejen til Kunderne}
Electroshoppen består idag af 40 butikker spredt rundt i hele landet. Det betyder at kunden har relativt kort til den nærmeste butik. Disse butikker har lige nu hver især deres egen lille lager som består af deres mest solgte varer. Derudover har de et centralt lager, hvorfra de kan sende produkter der ikke er plads til på deres små lagrer. 

Electroshoppen har planer om at udrydde behovet for dette lager, med deres webshop, som skal være i stand til at skabe en direkte kontakt med leverandøren. Denne webshop skal have nem adgang til de nyeste og bedste deals på elektronik. Den skal kunne tilgås gennem søgemaskiner som Pricerunner, Google, Bing osv. Det kræves, at de er konkurrencedygtige og istand til at udkonkurrere konkurrenter, når produkter bliver søgt på.

\section{Partner}
Electroshoppens nye systemlandskab skal gøre det nemmere og mere direkte at samarbejde med leverandører. Når kunder køber varer på hjemmesiden, køber de i virkeligheden varer direkte hos leverandøren, som så sender varerne til kunden. Dette sørger for at Electroshoppen kan spare både fragtomkostninger, speditionsinvolvering og lageromkostninger. Dette gør processtyring af nyligt bestilte varer, til de er leveret meget lettere og mere direkte. Dette hjælper også med at Electroshoppen kan byde på en dybt vareudbud, uden at have alle varer på lager.

\section{Resurser}
Det tætte samarbejde mellem Electroshoppen og dens leverandører gør det muligt at spare på ekstra lageromkostninger. Lagre i butikker kan gøres mindre, da ikke alle varer skal være fysisk tilstede i butikken. Dermed gøres inventaret større på to måder. På den ene side er butikkens lager mindre, og kan derfor udstille et mere bredt varetilbud. For det andet øges det ved at tilbyde at købe varer direkte hos leverandøren. Når en kunde vil have en vare, der ikke er i butikken, vil det være muligt at bestille varen hjem, eller sende den direkte til kunden fra leverandøren. Prisen kan holdes den samme, da Electroshoppen sparer egne fragt- og lageromkostninger. 

Medarbejdere behøver ikke længere at holde tæt øje på inventaret, for at bestille nye varer hjem. Medarbejdere kan bruge den samme hjemmeside for at købe varer ind, der kommer til at ligge til salg i butikkerne. Det gør det nemmere for medarbejdere, og reducerer oplæringstiden for nye medarbejdere.

\section{Aktiviteter}
ElectroMOS skal gøre det muligt for Electroshoppen, på simpel vis, at oprette og designe deres hjemmeside. ElectroMOS har et Content Management System, der gør det muligt via drag \& drop metoden at designe hjemmesider, der er beregnet til landingssider for kunder, kampagneside, produktsider og forside. 

Electroshoppens hjemmeside skal selv finde ud af, om de købte varer er på lager i en af butikkerne eller hos leverandøren, og sende ordren til det rette sted. Dette gør logistikken nemmere for Electroshoppen, fordi det ikke er så krævende for en medarbejder til at bestille varer 

til butikken eller en spedition, der først sender varer til butikken, der derefter sender det videre til kunden.

Sortimentsudvikling kan øges meget hurtigere på tværs af alle butikker, da varer kun skal oprettes i systemet ét sted, hvorefter butikker så har adgang til alle disse data. Sortimentet bliver dybere, da butikkerne ikke længere behøver at have alle produkterne på lageret for at tilbyde dem kunden.

Driften af lager vil blive nemmere med et system der køres på tværs af butikker og hjemmesiden. Systemet kan håndtere varer, bestillinger og logistikken mere automatisk. Ved at samle forskellige lagre, logistik og bestillingssystemer i et systemlandskab, vil den krævede uddannelse for nye medarbejder mindskes.

\section{Udgifter}
Lønomkostninger vil falde med det nye ElectroMOS, da alle medarbejder, i teorien, nemt kan håndterer deres butiks vareindkøb, og det kræver mindre know-how af medarbejderne at bruge ElectroMOS til at oprette kampagne-, landings- og forsider til hjemmesiden og dets produkter.

Det mere automatiserede køb af varer hos leverandøren vil sænke udgifterne til lagerbygninger, da der er mindre behov for disse. Butikkernes plads vil kunne blive udnyttet bedre, da behovet for at have mange varer på lager, vil være en mindre dominant i det nye systemlandskab. 

Butikkerne vil dermed have mere muligheder for at markedsføre et mere bredt vareudvalg og mere service i form af at sende varer hjem til kunden, ekstra køb som forsikringer og en moderne handelsoplevelse. 

\section{Indtægter}
Vareudbud vil være langt dybere med det nye system, da alle butikker udnytter vareudbud af det systemlandskab, der er fælles for hele kæden. Dette vil være en mere ensformig handelsoplevelse for kunden, ligegyldigt om kunden vælger at købe varer i en af butikkerne eller via hjemmesiden.

Hjemmesiden gør det også lettere at tilføje ekstra ydelser i form af produktforsikringer, udbringning eller installation af produkterne. Dette kan tilføjes både ved hjemmekøb via hjemmesiden eller direkte i butikken vha butikkens opsatte Ipads. Det forventes, at der vil være et øget køb af disse ekstra ydelser, da de bliver nemmere og mere bredt tilgængelig for kunden.

Desuden vil det være muligt for kunden at aflevere varer direkte i butikken til reparation eller via hjemmesiden. Da denne service bliver mere tydelig og strømlinet, forventes der dermed øgede indtægter via reparationer uden for garantiperioden.

\section{Opsummering}
Der er nu skrevet alle de væsentlige ting vha. Af Osterwalders model. Der er givet en beskrivelse af Electroshoppens forretningsmål, strategi og fået opbygget en struktur i virksomheden. Der er blevet beskrevet hvordan de forskellige sider til Electroshoppen kommer til at fungere. Den bliver simpel og nem at navigere rundt i, desuden vil det være nemt for ejerne at redigere indholdet, brugerne, informationer osv. Uden at de skal igennem en Webudvikler. Sammenhængen mellem indtægter, udgifter og leverandører er godt beskrevet og er et glimt af den fremragende strategi ElectroMOS har sat for Electroshoppen.