\newpage
\section{Læsevejledning}

Denne rapport er opdelt i flere centrale dele. Til at overskueliggøre opdelingen, kan der læses i denne sektion omkring det Til styring af rapportens detaljegrad, berørte emner m.m. er benyttet den udleveret projekthåndbog samt diverse andre kursusmaterialer.

I denne rapport vil der, ved afslutningen af hvert emne forefindes en kort opsummering, hvori den netop overståede sektions indhold gennemløbet kort. Således kan du, som læser, lettere holde det generelle overblik hele vejen igennem.


\subsection{Rapportens opdeling}
\subsubsection{Prolog}
Prologen indeholder flere punkter, der omhandler mere eksterne dele af projektet. I prologen kan der findes lige fra metoder til risici og resurser. Dette er derfor et punkt for sig, der omhandler nyttige punkter, som der med formål kan læses. Denne afsnit gennemgår den overordnede del, der ligger som grundsten for projektet. Såsom den brugte metode, projektforløb og prioritering.

\subsubsection{Inceptionsdokumentet}
Inceptionsdokumentet, er et dokument der indeholder meget kunderelateret information og begrunder beslutningen omkring den overordnete systemafgrænsningen som projektet går ud fra. Der eksisterer både overfladiske modeller, der omhandler det potentielle system – med megen information udeladt intentionelt. Derudover er der krav, der specificerer hvad systemet skal kunne tilbyde af service – såvel som krav der omhandler kvalitet og begrænsninger. Forretningsmodellen gennemgår begrundelser og eksistensgrundlag i forhold til kunden Elektroshoppen som organisation. Problemstillingen gennemgå domænen som projektet skal løse for Elektroshoppen.

\subsubsection{Elaborationsdokumentet}
Elaborationsdokumentet. Dette dokument indeholder en mere i dybdegående mængde informationer omkring systemet. Det indeholder opdaterede krav, samt analyse, design, implementering og test. Det indeholder derfor mange forskellige syn på systemet. Dog er implementering og test ikke så dominerende, da det kun eksisterer på et overfladisk niveau i elaborationsfasen. Derudover er der også en analyse af det organisationen. I denne del vil der blive fokuseret primært på strukturen.

\subsubsection{Epilog}
Epilog beskriver primært emner, der ligger ud over selve systemet. Dog indeholder den en overordnet konklusion, samt en referenceliste – og evaluering udarbejdet udefter gruppens oplevelse.

\subsubsection{Bilag}
Bilag indeholder informationer, der som bl. a. informerer om gruppens kontrakt og vejlederkontrakt. Én omkring milepæle og projektlog. Kildekode af systemet. Derudover, hvis tvivl på et ukendt ord kommer i læsningen på dette dokument, er ordbogen der med en forklaring.

\subsection{Farvekode}
Mange figurer har en farvekode der har til formål at synligøre de enkelte dele af diagrammerne. Den røde farve står for det overordnet system og viser systemafgransningen. Subsystemer og eksterne systemer er holdt i grønt. Pakker og brugsmønstre er gul. Klasser er repræsenteret med hvid baggrund.
