\chapter{Projektlog}
Projektlog for udarbejdelsen af ElectroMOS
 
\paragraph{Onsdag 15. Februar - Uge 7}
Møde med Henrik. Vi startede projektet op og afklaret, hvad der skal arbejdes med.
 
\paragraph{Onsdag 22. Februar - Uge 8}
Idag havde vi et kundemøde med hesehus, hvor vi fik stillet nogle gode spørgsmål som vi fik svar på. Disse spørgsmål vil være med til at danne fundamentet for vores projekt. Der er desuden også oprettet et projektlog, hvor vi skriver hvad vi har opnået og færdiggjort de kommende dage.
Vi sad alle til kundemødet og og stillede spørgsmål og fik svar på spørgsmål.
 
\paragraph{Torsdag 23. Februar - Uge 8}
Møde med Henrik. Planlægning af systemafgransning, og opbygning af de første udkast af brugsmønster og domainmodellen.
 
\paragraph{Tirsdag 28. Februar - Uge 9}
Vi er så stillet gået igang med at begynde på vores rapport, vi skriver inceptionsrapporten og skriver nogle fundamentale faser som Osterwaldersmodel og forretningsmodellen.
Alle i gruppen arbejder på inceptionsrapporten og Osterwaltermodels.
 
\paragraph{Onsdag 1. Marts - Uge 9}
Mens vi stadig er igang med at skrive og udfylde Osterwaldersmodel og forretningsmodellen. Har vi også oprettet et afsnit der hedder “Krav”, hvor vi skriver om vores funktionelle og ikke-funktionelle krav.
Vi delte gruppen i to, så Hamzat, Glenn og Kristian skriver videre på forretningsmodellen og Osterwalders model. Mens Marc, Thomas og Andreas er gået igang med det nye afsnit der hedder “Krav”.
 
\paragraph{Torsdag 2. Marts - Uge 9}
Hamzat, Glenn og Kristian er så småt ved at blive færdig med forretningsmodellen, og begynder at skrive en opsummering af afsnittet. Marc, Thomas og Andreas fylder stadig mere på “Krav” afsnittet, og er desuden også begyndt lidt på brugsmønsterdiagram og klassediagram.
 
\paragraph{Tirsdag 7. Marts - Uge 10}
Alle i gruppen samles igen og fylder lidt mere på “Krav” afsnittet, og begynder at skrive problemstillingen hvor vi kommer ind på produktoverblik og produktbeskrivelse. Der bliver også finpudset lidt på forretningsmodellen, så de andre også får at se hvad der er blevet lavet.
Møde med Henrik.
 
\paragraph{Onsdag 8. Marts - Uge 10}
Alle i gruppen går mere ind i dybden med “Problemstilling” afsnittet og begynder at skrive lidt om målet, og opsummeringen. Derefter går Marc, Hamzat og Andreas igang med at fylde lidt mere på produktoverblikket og produktbeskrivelsen. Kristian, Glenn og Thomas laver et bilag afsnit hvor de sætter alle diagrammer og billeder osv. ind.
 
\paragraph{Torsdag 9. Marts - Uge 10}
Alle i gruppen samles igen og skriver “Problemstilling” afsnittet færdigt. Alle i gruppen går derefter igang med at læse “Krav” afsnittet igennem og rette for fejl og lave det færdigt. Derefter går alle igang med at skrive om prioriteringen og opsummeringen af afsnittet og skriver det færdigt.
 
\paragraph{Torsdag 16. Marts - Uge 11}
Møde med Henrik
Alle i gruppen har samlet sig og skrevet en “Konklusion” på dokumentet og finpudset dokumentet for småfejl og osv.
 
\paragraph{Fredag 24. marts - Uge 12}
Hovedpunkterne for i dag består af at tilrette inceptionsdokumentet til den kritik vi har fået til reviewmødet d. 22 marts.
Andreas, Hamzat og Thomas har stået for at lave domænemodel, til udskiftning af det simple klassediagram der ligger i rapporten. Redigering af brugsmønstermodeller, prolog, MoSCoW prioriteringsafsnittet og “Krav” afsnittet.
Marc, Glenn og Kristian har stået for at lave Metoder, Resurser og Milepæle.
Derefter har alle i gruppen samlet sig igen og rettet dokumentet igennem for al kritikken der blev givet til review-mødet.
 
\paragraph{Onsdag 29. Marts - Uge 13}
Vi gør Inceptionsdokumentet færdigt.
Møde med Henrik hvor der bliver spurgt nogen afklarende spørgesmål ang. Inceptionsdokumentet, såsom indhold af detaljeret tidsplan.
Forberedelse for det næste kundemøde med Hesehus.
 
\paragraph{Torsdag 30. Marts - Uge 13}
Kundemøde med Hesehus
 
 
\paragraph{Torsdag 20. april - Uge 16}
Møde med Henrik.
Første møde i Elaborations fasen. vi startede ud med at planlægge hvordan vi skulle gribe denne fase an. vi startede ud med at tage udgangspunkt i vores domæne model og brugsmønster modeller valgt fra inceptions dokumentet.
Vi sluttede dagen med at komme til den konklusion at de fleste af vores modeller ikke matchede den mest optimale måde at lave dette program på, så nogen opdateringer skulle laves. vi er enige om at næste uge ikke bliver brugt til projekt, da vi har database eksamen.
Marc: Arbejde på analyse af vores diagrammer.
Kristian: Arbejde på analyse.
Andreas: Arbejde på analyse.
 
\paragraph{Torsdag 27. april - Uge 17}
Feedback på inception rapport.
Efter feedback, gennemgik vi resultater fra vores analyse og design vi havde arbejdet på hjemme fra. Første udkast af produktet er så småt under konstruktion.
Marc: Arbejde med opsætning af database og business lag.
Kristian: Arbejde med business lag og GUI.
 
\paragraph{Onsdag 3. maj - Uge 18}
vi gennemgik så småt resultatet af vores hjemmearbejde fra sidste gang. Database er på plads. Der mangler stadig arbejde på webshop og CMS. første udkast af rapporten  blev opsat i dag.
Marc: DB og webshop
Kristian: CMS og GUI.
Glenn: GUI.
Andreas: Widgets.
Hamzat: rapport opsætning.
 
\paragraph{Torsdag 4. maj - Uge 19}
Møde med Henrik
Vi satte deadline på vores opgaver til Tirsdag, så vi kunne forberede vores præsentation.
primært blev tiden brugt på at udvikle videre på en grundlæggende struktur af programmet
Marc: Webshop og CMS business plus analyse. 
Kristian: CMS business og GUI udvikling.
Glenn: CMS gui og widgets.
Andreas: CMS widgets.
Hamzat: rapport opsætning.
 
\paragraph{Fredag 5. maj - Uge 19}
Dagen var primært brugt til at færdiggøre vores deadline til Tirsdag. CMS var godt på vej til at være færdig til første fremvisning, webshoppen blev færdig til præsentation.
Design valget af database skal laves om så  det matcher vores nye løsninger.
Marc: DB og webshop.
Kristian: CMS.
Andreas: widget.
Glenn: GUI + widgets.
Hamzat: Rapportstruktur.
 
\paragraph{Søndag 7. maj - Uge 19}
Vi mødtes for at få det sidste kode på plads. det omhandlet primært design valg,  da webshoppen afhænger meget af hvilket data bliver gemt. 
Vi skulle være sikre på at der ikke var uenighed omkring hvordan data skulle gemmes.
Marc: færdiggøre analyse og design modeller for webshop
Kristian: færdiggøre analyse og design modeller for CMS
Glenn: finpudsning af GUI
Andreas: finpudsning af widgets
 
\paragraph{Tirsdag 9. maj - Uge 20}
Vi laved præsentation til Onsdag, opdateret alle diagrammer så de matchede de sidste kode implementeringer. sørget for koden var klar til at blive præsenteret til  fremlæggelsen.
Marc: sørget for webshop modeller plus kode + opsætning af powerpoint
Kristian: stod for modeller for CMS + demonstration af koden
Andreas: Stod for power point.
 
\paragraph{Onsdag 10. maj - Uge 20}
Præsentation: Marc, Kristian, Andreas og Glenn til stede.
brugte en time på at diskutere feedback, og planen for næste fase.
 
\paragraph{Torsdag 11. maj - Uge 20}
gennemgik det feedback fra præsentationen, og begyndte at diskutere de funktioner vi gerne vil implementere. Rapport udkast skulle færdiggøres til Mandag, så vi sørget for de dele vi havde arbejdet med var dokumenteret på en fornuftig måde. Rapporten havde ikke en struktur endnu, så vi brugte meget af dagen på at sætte rapport op.
Marc: Analyse dokumentation.
Kristian: Implementering.
Glenn: Design.
Andreas: Introduktion, Krav, og domæne model.
 
\paragraph{Fredag 12. maj - Uge 20}
Vi arbejde primært på de rapport udkast vi skulle aflevere mandag. sørget for det hele var dokumenteret som det burde.
samme arbejdsopgaver som den 11. maj
 
\paragraph{Tirsdag 16. maj - Uge 21}
Vi begyndte at implementere de opgaver vi havde snakket om Torsdag den 11. maj. dette var primært funktioner til at slette, opdatere og lave nye sider. dagen blev primært brugt på at implementere features.
Marc: opdatering af webshop med nye features.
Kristian: implementering af nye features.
Andreas: opdatering af widgets.
Glenn: opstilling af rapport til Latex + rapportskrivning.
 
\paragraph{Onsdag 17. maj - Uge 21}
Møde med Henrik.
Vi fik feedback på at vores kode struktur ikke var helt optimal. vi besluttede os derfor for at restrukturerer både webshop og CMS, så persistens laget håndterer alt SQL og database. Vi blev enige om at dette skulle være færdig til Mandag d. 22. maj. alt kodning skal være færdig 
på mandag. Vi arbejder hjemmefra torsdag, fredag og weekenden.
Marc: opdatering af webshop.
Kristian: opdatering af CMS.
Andreas: arbejder videre på widget.
Glenn: sørger for der stadig bliver arbejdet på rapport opbygning og struktur.
 
\paragraph{Mandag 22. maj - Uge 22}
Koden er blevet lavet færdig, og vi har uddelt de første opgaver til det resterende rapport.
Marc: sørger for Analysemodel section er lavet 100\% færdig, og laver design og implementering del for webshop.
Glenn: Laver struktur, og sørger for design er 100\% færdig.
Kristian: sørger for implementering og testning rapportskrivning.
Andreas: står for alt arbejde før analyse delen, primært krav og opdateret domæne model.
Hamzat: står for organisation og introduktion.
Dette skal være færdig  til på onsdag, hvor vi laver 2nd del af rapporten.
 
\paragraph{Onsdag 24. maj - Uge 22}
Den første del af rapporten er ved at være på plads, der er lavet det meste af analyse og implementering, hvorimod design delen hænger lidt. vi skal have lavet opfølgning på de sidste dele der mangler af rapporten og prøve at have alting færdigt inden Tirsdag. på den måde har vi tid til at rette rapporten igennem.
Marc: design og implementering + opdateret tidsplan og Metoder.
Kristian: Ret Analyse + testing. 
Andreas: Opdaterer domain model + krav.
 
\paragraph{Torsdag 25. maj - Uge 22}
Vi mødtes for at gennemgå vores resultat, og være sikker på alle arbejdede på noget. Andreas havde fri efter aftale med resten af gruppen. der blev ud arbejdet mere design for webshoppen + retning af analyse og implementering. der er overvejelser om at de mest aktive medlemmer skal lave resten, da der ikke er vist resultater fra de sidste 2 medlemmer.
Marc: Design af webshop + opdatering af introduktion af hele rapporten.
Kristian: Analyse og implementerings arbejde. så småt begyndt at arbejde med CMS design.
 
\paragraph{Fredag 26. maj - Uge 22}
videre arbejde for at nå deadline til Tirsdag. webshop design er på plads og krav er næsten færdig. Kristian holder fri, efter aftale med gruppen. Hamzat arbejder hjemmefra da han ikke har mulighed for transport. vi mangler stadig konklusion, design af CMS, og alt uden for elaboration dokumentet.
Marc: design af webshop + rette elaborations dokument.
Andreas: videre arbejde med krav.

 
\paragraph{Mandag 29. maj - Uge 23}
Kristian færdiggøre analyse. Marc har opdateret prolog fra Inceptions fasen, or er begyndt at arbejde med organisation og database beskrivelse. Andreas arbejder videre på opdatering af krav, brugsmønster og domæne model.Vi prøver at have rapporten færdig i løbet af i morgen, hvilket virker realistisk. så er der bare små rettelser tilbage.
Marc: Database, Orginisation, Prolog, hjælper Kristian når færdig.
Kristian: Afslut Design og Implementering.
Andreas: Krav og Ordbog.
Glenn: Design af CMS.

\paragraph{Tirsdag 30. maj - Uge 23}
Gruppen samlede primært for at opfølge på alle afsnit rapporten. den skal som udgangspunkt være klar til at blive rettet igennem i morgen, så vi minimum har to dage til at få styr på sidste. de sidste arbejdsfordelinger er blevet delt ud mellem de fremmødte medlemmer, og der er forventning om alle laver deres ansvarsområde.
Marc: Organisation, Konklussion, rettelser.
Krisitan: Analyse, Læsevejledning, test til latex, implementering.
Andreas: Krav, Forord, Evaluering.
Glenn: Abstract, Tidsplan, Design af CMS.

\paragraph{Onsdag 31. maj- Uge 23}
Skrivnings arbejdet var næsten færdigt. der manglede lige de sidste sektioner. Der blev primært arbejdet med opsætning, referancer, og redigering. de opmødte medlemmer er Marc, Kristian, Andreas og Glenn.
Alle arbejdede med det samme, da det primært bare var gennemgang af rapport

\paragraph{Torsdag 1. Juni - Uge 23}
Afslutning af rettelser, alle kigger det hele igennem en sidste gang før afleveringen i morgen. alt relativt ved at være færdigt.
Opmødt: Marc, Kristian, Andreas, Glenn.
Fredag d. 2 juni er til last minute problems, bare for at være sikre. plus aflevering.