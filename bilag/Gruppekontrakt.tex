\chapter{Gruppekontrakt}
\section{Fremmøde}
\paragraph{Som udgangspunkt mødes gruppen følgende dage:}
Tirsdag fra 12:00-16:00
Onsdag fra 12:00-16:00
Torsdag fra 12:00-16:00
Der er forventninger til at alle gruppens medlemmer bruger 8-12 timer om ugen med projekt relateret arbejde.
12+ timer er selvfølgelig tilladt.

\paragraph{Ændringer af tidspunkter:}
Mødetider er ændret hvis undervisning konflikter med aftalte tider.
Mødetider kan ændres hvis flertal fra gruppen er enige.
Flere mødetider kan tilføjes hvis flertal stemmer for det.
Mødetider kan fjernes hvis flertal stemmer for det.

\paragraph{Manglende fremmøde:}
Der er forventning til at alle gruppens medlemmer møder til de aftalte                 tidspunkter.
Hvis man er forhindret fremmøde grundet sygdom/nødstilfælde bedes man informere gruppen om det, så hurtigt som muligt.
Hvis man er forhindret fremmøde grundet ferie/arbejde/andet skal man informere gruppen mindst en uge før.
Gruppen kan med flertal bestemme en undskyldning for manglende fremmøde ikke er godkendt, og vil derfor tælle som at være fraværende.
Hvis et Gruppemedlem er fraværende uden grund, skal den fraværende gruppemedlem medbringe kage til hele gruppen ved næste fremmøde.
Hvis et Gruppemedlem er fraværende over længere periode, vil der blive afholdt en samtale, angående grunden til udeblivelsen.
Længere fravær vil blive informeret til vejleder.




\section{Forventninger til gruppeindsats}
Det forventes af alle i gruppen gør en indsats for projektet.
Det optimale mål for arbejdsfordelingen er at alle har haft en lige så stor del i hele projektet og rapporten.

\section{Gruppestruktur}
Gruppens størrelse er på 6 medlemmer.
En af gruppens medlemmer bliver tildelt rollen som projektleder.
Gruppeleder tager den endelige beslutning i projektgruppen hvis der forekommer uenighed.
Hvis der er indbyrdes problemer i gruppen, skal det først tages op med projektlederen.
Gruppelederen kan blive udskiftet hvis de resterende gruppemedlemmer, med følgende grundlag.
Hvis gruppelederen laver beslutninger uden at tage det op med gruppen.
Hvis gruppelederen møder ikke op til aftalte mødetider.
Hvis gruppelederen har fravær fra undervisningen over en længere periode.
Som udgangspunkt skal man have fremhævet sin bekymring til gruppen og gruppelederen inden udskiftning, så der er en chance for forbedring.
Til projektrelateret møder skal der være en Ordfører og en Skribent til  stede.

\paragraph{Ordfører: }
Ordfører går på skift til hvert møde. 
Alle gruppens medlemmer skal have prøvet at være ordfører til mindst et møde.
Gruppens medlemmer kan fratages titel og ansvar som ordføre, hvis gruppen siger god for det. 
Ordfører står for at holde møderne relevant til projektgruppens spørgsmål.
Ordføreren sørger for at få booket mødet med vejlederen.
Ordføreren sørger for der er lokale booket til vejledermøder.
Ved sygdom/fravær bytter den nuværende ordfører plads med næste mødes ordfører.


\paragraph{Skribent: }
Skribent går på skift til hvert møde.
Alle gruppens medlemmer skal have prøvet at være skribent til mindst et møde. 
Man kan ikke fratages fra skribentAnsvarlighed.
Skribent skal fremlægge sin logbog mindst 3-4 dage efter mødet.
Skribenten skal fører en logbog for hvertmøde, mere information i punkt 3.2.
Ved sygdom/fravær bytter den nuværende skribent plads med næste mødes skribent.

\paragraph{Beslutninger på gruppen}
Beslutninger i forbindelse med projektet, skal have flertal fra gruppens medlemmer for at kunne blive integreret i det afsluttende aflevering.
Hvis der er uenighed skal gruppens leder tage endelige beslutning.
Hvis der er uenighed se punkt 2.1.2.1.


\section{Dokumentation}

\paragraph{Rapport}
Rapporten skal formuleres på dansk.
Alle gruppens medlemmer skal deltage i rapportskrivning.
Hvis et gruppemedlem nægter at deltage i rapportskrivning, vil det blive rapporteret som at være fraværende.
Alle gruppens aktive medlemmer skal have godkendt rapporten inden aflevering.

\paragraph{Logbog}
Logbogen skal være lavet på dansk.
Logbogen skal indeholde referat af det nyeste møde.
Logbogen skal indeholde nuværende status af projektet.
Logbogen skal indeholde en plan/milestones til næste møde.
Logbogen skal godkendes af gruppen, inden den kan bruges til bilag.

\paragraph{Kode og Bilag}
Alt kode skal være skrevet i Java og postgreSQL.
Grafiske brugergrænseflader skal holdes i JavaFX.
Alle gruppens aktive medlemmer, skal være enige om kodens struktur inden aflevering.
Der skal holdes kommentarer på det kode der bliver lavet til projektet.