


\section{ Organisatoriske design}


I denne sektion er der primært taget fokus i hvordan elektroshoppens business design er opstillet. Der kommes med konkrete eksempler på hvordan de Elektroshoppens business strukturer kan fungere med en central webshop. og hvad den bedste løsning eventuelt kunne være for at strukturer sit firma til webshoppen som et centralt knudepunkt mellem butikkerne.


\subsection{Organisations struktur}
Electroshoppen er baseret på en typisk struktur, som giver mening for et elektronik varehus. Den struktur er oftest ejet af en bestyrelse der ejer primære aktier i firmaet. Denne bestyrelser tager ofte beslutninger om hvilket retning firmaet skal gå i. Disse valg bliver ofte udviklet fra et hovedkontor, der laver analyser, ud fra hvad der kan optjene flest penge for firmaet. fra hovedkontoret er næste skridt ned til en supervisor. denne supervisor tager ud til butikker og sørger for de overholder reglerne fra hovedkontoret. Modellen bliver så spredt ud i de forskellige branches eller butikker, hvor strukturen ses opdelt efter, at der er en branch manager. Branch manageren sørger normalt for at denne specifikke butik kører så optimalt som muligt, og holder de guidelines som hovedkontoret nu har bestemt for butikkerne. hver butik er opdelt i afsnit, som står for hver sin arbejds opgave. Disse arbejdsopgaver kunne være RMA, salg, Varehåndtering og Support, hver afdeling har en ansvarlig, som kommunikere med branch manager.
 
På denne måde er Elektroshoppen sat efter en mere vertikal struktur, hvor i at der er en tydelig chain-of-command ned igennem strukturen. Man ser tit ikke CEO gå direkte ned til butikkerne for at sørger for at de kører som de skal, derimod bruger han supervisors der inspicerer butikkerne, og sørger for de kører efter standard.

Denne standard er bestemt gennem øverste ledelse, hvor alle butikker primært skal rette sig efter denne formular. Dette giver også meget mening når webshoppen skal have meget sammenspil med alle butikker i landet. Derfor rettes der meget mod en mekanisk struktur med fokus på at have en integreret business model i alle deres forretninger. 
 
Det giver dog mening for Elektroshoppen, at lade hver eneste branch styre hvilket vare de oftest har på lager, hvilket kampagner der skal køres for de enkelte butikker, og hvordan de skal håndtere deres kunder, da de enkelte butikker ofte bedst selv kender deres klientel. Opgaver som struktur for medarbejde træning, varer udvalg, lønninger, store beslutninger. Vil nok ses som værende en opgave for det centrale hovedkontor. Derfor vil Elektroshoppen nok primært være et centraliseret firma, med noget selvstændighed i de enkelte branches.

\begin{figure}[H]
	\includegraphics[width=\linewidth]{elaborationsdokumentet/figurer/Organisation.png}
    \caption{Organisations struktur over Elektroshoppen}
    \label{struktur}
\end{figure}
 
Som nævnt før giver det mening for Elektroshoppen at være meget centraliseret og mekanisk, da webshoppen skal til at være et samlingspunkt for alle deres butikker. Derfor skal butikkerne ikke varierer alt for meget fra hinanden. Kunden kan være sikker på at lige meget hvor de tilgår hjemmesiden, kan de altid komme ned i en butik, og forvente de får alt det de er blevet lovet. Det er ligemeget hvor i landet butikken ligger. Hvis hver butik havde hver deres måde at køre butikken på, Kunne man ikke garantere noget igennem webshoppen ud at risikere de lige præcis kan udfører det stykke arbejde i den butik.  Webshoppen giver dog mulighed for at hver eneste butik kan tilpasse sit varelager ud fra hvad folk bestiller mest. Og på den måde stadig have få beslutning der kan blive bestemt af den enkelte branch manager. Derfor kan man ikke sige det er 100\% mekanisk, da der også ses features fra en mere organisk business struktur. Alt teori er fundet igennem \cite{organisation} da kilden er pålidelig, og forklarede det bedst.

\subsection{Opsummering}
I dette afsnit, blev  der identificeret Elektroshoppens businessmodel, udfra analyser af case modellen, er der blevet identificeret at elektroshoppens struktur ses mest som en mekanisk model. Dette betyder at Elektroshoppen arbejde primært ud fra en centralt hovedkontor, dette giver rigtig god mulighed for hver eneste butik kan fungerer synkront med webshoppen, da der nemt kan indføres ny politik der matcher webshoppens rammer.

